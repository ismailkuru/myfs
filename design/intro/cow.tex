\section{Копирование при записи}

Как упоминалось ранее, при использовании COW при записи новых данных или
изменении старых файловая система алоцирует новое место на диске и создает
новую версию. Одним из достоинств использования такого подхода является
возможность эффективной реализации снимков состояния (snapshot), в то время как
файловым системам не использующим COW такая функциональность не доступна.

\subsection{Снимки состояния}

При использовании COW реализация снимков состояния является в некоторой степени
простой. Так как при изменении COW создает новую копию данных, все что нужно для
реализации снимков - это сохранить ссылку на копию данных и запретить
переиспользование места, занятого этой копией данных. Если смотреть на всю
файловую систему целиком как на одно единственное дерево, то при обновлении
необходимо сначала найти нужный узел дерева, затем создать копию этого узла и
изменить ее соответствующим образом, после чего необходимо обновить ссылку на
этот узел в родительском узле. Обновление ссылки в родительском узле так же
является изменением, соответственно, требуется опять же создать копию и изменить
копию. Процесс копирования заканчивается, когда файловая система создаст новый
корень дерева. Таким образом старый и новый корень дерева будут представлять
версии файловой системы до и после изменения соответственно, а чтобы создать
снимок состояния файловой системе достаточно сохранить текущий корень.

Естественно возможность создания снимков состояния не была бы важной если бы
снимкам не было практического применения. Снимки состояния позволяют получить
доступ к консистентному стабильному состоянию файловой системы. Эту возможность
можно использовать, чтобы сохранить состояние файловой системы перед какими-либо
изменениями, например, такими как установка или удаление приложений. Если по
каким-то причинам потребуется откатить изменения, то для этого можно
использовать созданный снимок.

Другое применения снимков состояния - это резервное копирование. Создать
резервную копию данных, когда данные постоянно изменяются довольно
проблематично. Поэтому снимки состояния, так как они предоставляют доступ к
созраненному состоянию файловой системы позволяют упростить задачу. Все что
требуется это создать снимок и создать резервную копию из этого снимка.

\subsection{Транзакции}

Другое достоинство использования COW - это обеспечение транзакционности. Ссылка
на обновленную версию данных или новые данные обновляется когда данные уже
записаны на диск, таким образом на диске не может быть записана только часть
транзакции. Транзакция либо применяется полностью, когда файловая система
создает новый корень, либо не применяется вовсе.
